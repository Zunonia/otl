\section{\module{sybasect} --- Interface to Sybase-CT library}

\declaremodule[sybasect]{extension}{sybasect}

\modulesynopsis{Interface to Sybase-CT library.}

This is not a complete reference to the Sybase CT library.  Sybase
produce excellent documentation which fully describes the use of the
CT library.

This section describes how to access the Sybase CT library while
using this wrapper module.

The \module{sybasect} extension contains the following:

\subsection{Types}

\begin{datadesc}{ContextType}
The type of \code{CS_CONTEXT} objects which wrap the Sybase
\code{CS_CONTEXT} structure pointer.
\end{datadesc}

\begin{datadesc}{ConnectionType}
The type of \code{CS_CONNECTION} objects which wrap the Sybase
\code{CS_CONNECTION} structure pointer.
\end{datadesc}

\begin{datadesc}{CommandType}
The type of \code{CS_COMMAND} objects which wrap the Sybase
\code{CS_COMMAND} structure pointer.
\end{datadesc}

\begin{datadesc}{BlkDescType}
The type of \code{CS_BLKDESC} objects which wrap the Sybase
\code{CS_BLKDESC} structure pointer.
\end{datadesc}

\begin{datadesc}{DataFmtType}
The type of \code{CS_DATAFMT} objects which wrap the Sybase
\code{CS_DATAFMT} structure.
\end{datadesc}

\begin{datadesc}{IODescType}
The type of \code{CS_IODESC} objects which wrap the Sybase
\code{CS_IODESC} structure.
\end{datadesc}

\begin{datadesc}{ClientMsgType}
The type of \code{CS_CLIENTMSG} objects which wrap the Sybase
\code{CS_CLIENTMSG} structure.
\end{datadesc}

\begin{datadesc}{ServerMsgType}
The type of \code{CS_SERVERMSG} objects which wrap the Sybase
\code{CS_SERVERMSG} structure.
\end{datadesc}

\begin{datadesc}{DataBufType}
The type of data buffers for sending and receiving data to and from
Sybase.  The type of object returned by \code{DataBuf('hello')}.
\end{datadesc}

\begin{datadesc}{NumericType}
The type used to store Sybase \code{CS_NUMERIC} and \code{CS_DECIMAL}
data values.
\end{datadesc}

\begin{datadesc}{DateTimeType}
The type used to store Sybase \code{CS_DATETIME} and
\code{CS_DATETIME4} data values.
\end{datadesc}

\begin{datadesc}{MoneyType}
The type used to store Sybase \code{CS_MONEY} and \code{CS_MONEY4}
data values.
\end{datadesc}

\subsection{Functions}

\begin{funcdesc}{set_global_ctx}{ctx}
The \module{sybasect} module uses a \code{CS_CONTEXT} structure
internally for conversions and calculations.  You must allocate a
context via \function{cs_ctx_alloc()} and establish the internal
context using this function.
\end{funcdesc}

\begin{funcdesc}{set_debug}{file}
Directs all debug text to the object passed in the \var{file}
argument.  The \var{file} argument must be either \code{None} or an
object which has \method{write(data)} and \method{flush()} methods.

The function returns the previous debug file.  The default file is
\code{None}.

Setting a debug file does not enable debug messages.  To produce debug
messages you need to set the \var{debug} member of a context,
connection, command, etc.
\end{funcdesc}

\begin{funcdesc}{cs_ctx_alloc}{\optional{version \code{= CS_VERSION_100}}}
Calls the Sybase-CT \function{cs_ctx_alloc()} function:

\begin{verbatim}
result = cs_ctx_alloc(version, &ctx);
\end{verbatim}

Returns a tuple containing the Sybase result code and a new instance
of the \class{CS_CONTEXT} class constructed from the \var{ctx} value
returned by \function{cs_ctx_alloc()}.  \code{None} is returned as the
\class{CS_CONTEXT} object if the result code is not \code{CS_SUCCEED}.
\end{funcdesc}

\begin{funcdesc}{cs_ctx_global}{\optional{version \code{= CS_VERSION_100}}}
Calls the Sybase-CT \function{cs_ctx_global()} function:

\begin{verbatim}
result = cs_ctx_global(version, &ctx);
\end{verbatim}

Returns a tuple containing the Sybase result code and a new instance
of the \class{CS_CONTEXT} class constructed from the \var{ctx} value
returned by \function{cs_ctx_global()}.  \code{None} is returned as the
\class{CS_CONTEXT} object if the result code is not \code{CS_SUCCEED}.
\end{funcdesc}

\begin{funcdesc}{DataBuf}{obj}
Return a new instance of the \class{DataBuf} class.  The \var{obj}
argument is used to initialise the \class{DataBuf} object.

For all types of \var{obj} other than \class{CS_DATAFMT} a buffer will
be initialised which contains a single value.

When \var{obj} is a \class{CS_DATAFMT} object an empty buffer will be
created according to the attributes of the \class{CS_DATAFMT} object.
It is most common to create and bind a buffer in a single operation
via the \method{ct_bind()} method of the \class{CS_COMMAND} class.

For example, the following code creates a set of buffers for
retrieving 16 rows at a time.  Note that it is your responsibility to
ensure that the buffers are not released until they are no longer
required.

\begin{verbatim}
status, num_cols = cmd.ct_res_info(CS_NUMDATA)
if status != CS_SUCCEED:
    raise 'ct_res_info'
bufs = []
for i in range(num_cols):
    status, fmt = cmd.ct_describe(i + 1)
    if status != CS_SUCCEED:
        raise 'ct_describe'
    fmt.count = 16
    status, buf = cmd.ct_bind(i + 1, fmt)
    if status != CS_SUCCEED:
        raise 'ct_bind'
    bufs.append(buf)
\end{verbatim}
\end{funcdesc}

\begin{funcdesc}{numeric}{obj \optional{, precision \code{= -1}} \optional{, scale \code{= -1}}}
Return a new instance of the \class{Numeric} class.

The \var{obj} argument can accept any of the following types;
\code{IntType}, \code{LongType}, \code{FloatType}, \code{StringType},
or \code{NumericType}.

If greater than or equal to zero the \var{precision} and \var{scale}
arguments are used as the \member{precision} and \member{scale}
attributes of the \code{CS_DATAFMT} which is used in the Sybase
\function{cs_convert()} function to create the new \class{NumericType}
object.  The default values for these arguments depends upon the type
being converted to \class{NumericType}.

\begin{longtable}{l|l|l}
Type & \var{precision} & \var{scale} \\
\hline
\code{IntType}     & \code{16} & \code{0} \\
\code{LongType}    & \# of digits in \code{str()} conversion & \code{0} \\
\code{FloatType}   & \code{CS_MAX_PREC} & \code{12} \\
\code{StringType}  & \# digits before decimal point & \# digits after decimal point \\
\code{NumericType} & input precision & input scale \\
\end{longtable}
\end{funcdesc}

\begin{funcdesc}{money}{obj \optional{, type \code{= CS_MONEY_TYPE}}}
Return a new instance of the \class{Money} class.

The \var{obj} argument can accept any of the following types;
\code{IntType}, \code{LongType}, \code{FloatType}, \code{StringType},
or \code{MoneyType}.

Passing \code{CS_MONEY4_TYPE} in the \var{type} argument will create a
\code{smallmoney} value instead of the default \code{money}.
\end{funcdesc}

\begin{funcdesc}{datetime}{str \optional{, type \code{= CS_DATETIME_TYPE}}}
Return a new instance of the \class{DateTime} class.

The \var{str} argument must be a string.

Passing \code{CS_DATETIME4_TYPE} in the \var{type} argument will
create a \code{smalldatetime} value instead of the default
\code{datetime}.
\end{funcdesc}

\begin{funcdesc}{sizeof_type}{type_code}
Returns the size of the type identified by the Sybase type code
specified in the \var{type_code} argument.

The values expected for the \var{type_code} argument things like;
\code{CS_CHAR_TYPE}, \code{CS_TINYINT_TYPE}, etc.
\end{funcdesc}

\begin{funcdesc}{CS_DATAFMT}{}
Return a new instance of the \class{CS_DATAFMT} class.  This is used
to wrap the Sybase \code{CS_DATAFMT} structure.
\end{funcdesc}

\begin{funcdesc}{CS_IODESC}{}
Return a new instance of the \class{CS_IODESC} class.  This is used
to wrap the Sybase \code{CS_IODESC} structure.
\end{funcdesc}

\begin{funcdesc}{CS_LAYER}{msgnumber}
Return the result of applying the Sybase \code{CS_LAYER} macro to the
\var{msgnumber} argument.
\end{funcdesc}

\begin{funcdesc}{CS_ORIGIN}{msgnumber}
Return the result of applying the Sybase \code{CS_ORIGIN} macro to the
\var{msgnumber} argument.
\end{funcdesc}

\begin{funcdesc}{CS_SEVERITY}{msgnumber}
Return the result of applying the Sybase \code{CS_SEVERITY} macro to the
\var{msgnumber} argument.
\end{funcdesc}

\begin{funcdesc}{CS_NUMBER}{msgnumber}
Return the result of applying the Sybase \code{CS_NUMBER} macro to the
\var{msgnumber} argument.
\end{funcdesc}

\subsection{CS_CONTEXT Objects}

Calling the \function{cs_ctx_alloc()} or \function{cs_ctx_global()}
function will create a \class{CS_CONTEXT} object.  When the
\class{CS_CONTEXT} object is deallocated the Sybase
\function{cs_ctx_drop()} function will be called for the context.

\class{CS_CONTEXT} objects have the following interface:

\begin{memberdesc}[CS_CONTEXT]{debug}
An integer which controls printing of debug messages to the debug file
established by the \function{set_debug()} function.  The default value
is zero.
\end{memberdesc}

\begin{methoddesc}[CS_CONTEXT]{debug_msg}{msg}
If the \member{debug} member is non-zero the \var{msg} argument will
be written to the debug file established by the \function{set_debug()}
function.
\end{methoddesc}

\begin{methoddesc}[CS_CONTEXT]{cs_config}{action, property \optional{, value}}
Configures, retrieves and clears properties of the \texttt{comn}
library for the context.

When \var{action} is \code{CS_SET} a compatible \var{value} argument
must be supplied and the method returns the Sybase result code.  The
Sybase-CT \function{cs_config()} function is called like this:

\begin{verbatim}
/* bool property value */
status = cs_config(ctx, CS_SET, property, &bool_value, CS_UNUSED, NULL);

/* int property value */
status = cs_config(ctx, CS_SET, property, &int_value, CS_UNUSED, NULL);

/* string property value */
status = cs_config(ctx, CS_SET, property, str_value, CS_NULLTERM, NULL);

/* locale property value */
status = cs_config(ctx, CS_SET, property, locale, CS_UNUSED, NULL);

/* callback property value */
status = cs_config(ctx, CS_SET, property, cslib_cb, CS_UNUSED, NULL);
\end{verbatim}

When \var{action} is \code{CS_GET} the method returns a tuple
containing the Sybase result code and the property value.  The
Sybase-CT \function{cs_callback()} function is called like this:

\begin{verbatim}
/* bool property value */
status = cs_config(ctx, CS_GET, property, &bool_value, CS_UNUSED, NULL);

/* int property value */
status = cs_config(ctx, CS_GET, property, &int_value, CS_UNUSED, NULL);

/* string property value */
status = cs_config(ctx, CS_GET, property, str_buff, sizeof(str_buff), &buff_len);
\end{verbatim}

When \var{action} is \code{CS_CLEAR} the method clears the property
and returns the Sybase result code.  The Sybase-CT
\function{cs_callback()} function is called like this:

\begin{verbatim}
status = cs_config(ctx, CS_CLEAR, property, NULL, CS_UNUSED, NULL);
\end{verbatim}

The recognised properties are:

\begin{longtable}{l|l}
\var{property} & type \\
\hline
\code{CS_EXTERNAL_CONFIG} & \code{bool} \\
\code{CS_EXTRA_INF}       & \code{bool} \\
\code{CS_NOAPI_CHK}       & \code{bool} \\
\code{CS_VERSION}         & \code{int} \\
\code{CS_APPNAME}         & \code{string} \\
\code{CS_CONFIG_FILE}     & \code{string} \\
\code{CS_LOC_PROP}        & \code{locale} \\
\code{CS_MESSAGE_CB}      & \code{function} \\
\end{longtable}

For an explanation of the property values and get/set/clear semantics
please refer to the Sybase documentation.
\end{methoddesc}

\begin{methoddesc}[CS_CONTEXT]{ct_callback}{action, type \optional{, cb_func \code{= None}}}
Installs, removes, or queries current Sybase callback function.  This
is only available when the \module{sybasect} module has been compiled
without the \code{WANT_THREADS} macro defined in \texttt{sybasect.h}.

When \code{CS_SET} is passed as the \var{action} argument the
Sybase-CT \function{ct_callback()} function is called like this:

\begin{verbatim}
status = ct_callback(ctx, NULL, CS_SET, type, cb_func);
\end{verbatim}

The \var{cb_func} argument is stored inside the \code{CS_CONTEXT}
object.  Whenever a callback of the specified type is called by the
Sybase CT library, the \module{sybasect} wrapper locates the
corresponding \code{CS_CONTEXT} object and calls the stored function.

If \code{None} is passed in the \var{cb_func} argument the callback
identified by \var{type} will be removed.  The Sybase result code is
returned.

When \var{action} is \code{CS_GET} the Sybase-CT
\function{ct_callback()} function is called like this:

\begin{verbatim}
status = ct_callback(ctx, NULL, CS_GET, type, &cb_func);
\end{verbatim}

The return value is a two element tuple containing the Sybase result
code and the current callback function.  When the Sybase result code
is not \code{CS_SUCCEED} or there is no current callback, the returned
function will be \code{None}.

The \var{type} argument identifies the callback function type.
Currently only the following callback functions are supported.

\begin{longtable}{l|l}
\var{type} & callback function arguments \\
\hline
\code{CS_CLIENTMSG_CB} & \code{ctx, conn, msg} \\
\code{CS_SERVERMSG_CB} & \code{ctx, conn, msg} \\
\end{longtable}

The following will allocate and initialise a CT library context then
will install a callback.

\begin{verbatim}
from sybasect import *

def ctlib_server_msg_handler(conn, cmd, msg):
    return CS_SUCCEED

status, ctx = cs_ctx_alloc()
if status != CS_SUCCEED:
    raise CSError(ctx, 'cs_ctx_alloc')
if ctx.ct_init(CS_VERSION_100):
    raise CSError(ctx, 'ct_init')
if ctx.ct_callback(CS_SET, CS_SERVERMSG_CB,
                   ctlib_server_msg_handler) != CS_SUCCEED:
    raise CSError(ctx, 'ct_callback')
\end{verbatim}
\end{methoddesc}

\begin{methoddesc}[CS_CONTEXT]{cs_loc_alloc}{}
Allocates a new \class{CS_LOCALE} object which is used to control
locale settings.  Calls the Sybase-CT \function{cs_loc_alloc()}
function like this:

\begin{verbatim}
status = cs_loc_alloc(ctx, &locale);
\end{verbatim}

Returns a tuple containing the Sybase result code and a new instance
of the \class{CS_LOCALE} class constructed from the \var{locale}
returned by \function{cs_loc_alloc()}.  \code{None} is returned as the
\class{CS_LOCALE} object when the result code is not \code{CS_SUCCEED}.
\end{methoddesc}

\begin{methoddesc}[CS_CONTEXT]{ct_con_alloc}{}
Allocates a new \class{CS_CONNECTION} object which is used to connect
to a Sybase server.  Calls the Sybase-CT \function{ct_callback()}
function like this:

\begin{verbatim}
status = ct_con_alloc(ctx, &conn);
\end{verbatim}

Returns a tuple containing the Sybase result code and a new instance
of the \class{CS_CONNECTION} class constructed from the \var{conn}
returned by \function{ct_con_alloc()}.  \code{None} is returned as the
\class{CS_CONNECTION} object when the result code is not
\code{CS_SUCCEED}.
\end{methoddesc}

\begin{methoddesc}[CS_CONTEXT]{ct_config}{action, property \optional{, value}}
Sets, retrieves and clears properties of the context object

When \var{action} is \code{CS_SET} a compatible \var{value} argument
must be supplied and the method returns the Sybase result code.  The
Sybase-CT \function{ct_config()} function is called like this:

\begin{verbatim}
/* int property value */
status = ct_config(ctx, CS_SET, property, &int_value, CS_UNUSED, NULL);

/* string property value */
status = ct_config(ctx, CS_SET, property, str_value, CS_NULLTERM, NULL);
\end{verbatim}

When \var{action} is \code{CS_GET} the method returns a tuple
containing the Sybase result code and the property value.  The
Sybase-CT \function{ct_callback()} function is called like this:

\begin{verbatim}
/* int property value */
status = ct_config(ctx, CS_GET, property, &int_value, CS_UNUSED, NULL);

/* string property value */
status = ct_config(ctx, CS_GET, property, str_buff, sizeof(str_buff), &buff_len);
\end{verbatim}

When \var{action} is \code{CS_CLEAR} the method clears the property
and returns the Sybase result code.  The Sybase-CT
\function{ct_callback()} function is called like this:

\begin{verbatim}
status = ct_config(ctx, CS_CLEAR, property, NULL, CS_UNUSED, NULL);
\end{verbatim}

The recognised properties are:

\begin{longtable}{l|l}
\var{property} & type \\
\hline
\code{CS_LOGIN_TIMEOUT} & \code{int} \\
\code{CS_MAX_CONNECT}   & \code{int} \\
\code{CS_NETIO}         & \code{int} \\
\code{CS_NO_TRUNCATE}   & \code{int} \\
\code{CS_TEXTLIMIT}     & \code{int} \\
\code{CS_TIMEOUT}       & \code{int} \\
\code{CS_VERSION}       & \code{int} \\
\code{CS_IFILE}         & \code{string} \\
\code{CS_VER_STRING}    & \code{string} \\
\end{longtable}

For an explanation of the property values and get/set/clear semantics
please refer to the Sybase documentation.
\end{methoddesc}

\begin{methoddesc}[CS_CONTEXT]{ct_exit}{\optional{option \code{= CS_UNUSED}}}
Calls the Sybase \function{ct_exit()} function like this:

\begin{verbatim}
status = ct_exit(ctx, option);
\end{verbatim}

Returns the Sybase result code.
\end{methoddesc}

\begin{methoddesc}[CS_CONTEXT]{ct_init}{\optional{version \code{= CS_VERSION_100}}}
Initialises the context object and tells the CT library which version
of behaviour is expected.  This method must be called immediately
after creating the context.  The Sybase \function{ct_init()} function
is called like this:

\begin{verbatim}
status = ct_init(ctx, version);
\end{verbatim}

Returns the Sybase result code.
\end{methoddesc}

\begin{methoddesc}[CS_CONTEXT]{cs_ctx_drop}{}
Calls the Sybase \function{cs_ctx_drop()} function like this:

\begin{verbatim}
status = cs_ctx_drop(ctx);
\end{verbatim}

Returns the Sybase result code.

This method will be automatically called when the \class{CS_CONTEXT}
object is deleted.  Applications do not need to call the method.
\end{methoddesc}

\begin{methoddesc}[CS_CONTEXT]{cs_diag}{operation \optional{, \ldots}}
Manage Open Client/Server error messages for the context.

When \var{operation} is \code{CS_INIT} a single argument is accepted
and the Sybase result code is returned.  The Sybase
\function{cs_diag()} function is called like this:

\begin{verbatim}
status = cs_diag(ctx, CS_INIT, CS_UNUSED, CS_UNUSED, NULL);
\end{verbatim}

When \var{operation} is \code{CS_MSGLIMIT} two additional arguments
are expected; \var{type} and \var{num}.  The Sybase result code is
returned.  The Sybase \function{cs_diag()} function is called like
this:

\begin{verbatim}
status = cs_diag(ctx, CS_MSGLIMIT, type, CS_UNUSED, &num);
\end{verbatim}

When \var{operation} is \code{CS_CLEAR} an additional \var{type}
argument is accepted and the Sybase result code is returned.  The
Sybase \function{cs_diag()} function is called like this:

\begin{verbatim}
status = cs_diag(ctx, CS_CLEAR, type, CS_UNUSED, NULL);
\end{verbatim}

When \var{operation} is \code{CS_GET} two additional arguments are
expected; \var{type} which currently must be \code{CS_CLIENTMSG_TYPE},
and \var{index}.  A tuple is returned which contains the Sybase result
code and the requested \class{CS_CLIENTMSG} message.  \code{None} is
returned as the message object when the result code is not
\code{CS_SUCCEED}.  The Sybase \function{cs_diag()} function is called
like this:

\begin{verbatim}
status = cs_diag(ctx, CS_GET, type, index, &msg);
\end{verbatim}

When \var{operation} is \code{CS_STATUS} an additional \var{type}
argument is accepted.  A tuple is returned which contains the Sybase
result code and the number of messages available for retrieval.  The
Sybase \function{cs_diag()} function is called like this:

\begin{verbatim}
status = cs_diag(ctx, CS_STATUS, type, CS_UNUSED, &num);
\end{verbatim}

The following will retrieve and print all messages from the context.

\begin{verbatim}
def print_msgs(ctx):
    status, num_msgs = ctx.cs_diag(CS_STATUS, CS_CLIENTMSG_TYPE)
    if status == CS_SUCCEED:
        for i in range(num_msgs):
            status, msg = ctx.cs_diag(CS_GET, CS_CLIENTMSG_TYPE, i + 1)
            if status != CS_SUCCEED:
                continue
            for attr in dir(msg):
                print '%s: %s' % (attr, getattr(msg, attr))
    ctx.cs_diag(CS_CLEAR, CS_CLIENTMSG_TYPE)
\end{verbatim}
\end{methoddesc}


\subsection{CS_LOCALE Objects}

\class{CS_LOCALE} objects are a wrapper around the Sybase
\code{CS_LOCALE} structure.  The objects are created by calling the
\method{cs_loc_alloc()} method of a \class{CS_CONTEXT} object.

They have the following interface:

\begin{methoddesc}[CS_LOCALE]{cs_dt_info}{action, type \optional{, \ldots}}
Sets or retrieves datetime information of the locale object

When \var{action} is \code{CS_SET} a compatible \var{value} argument
must be supplied and the method returns the Sybase result code.  The
Sybase-CT \function{cs_dt_info()} function is called like this:

\begin{verbatim}
status = cs_dt_info(ctx, CS_SET, locale, type, CS_UNUSED,
                    &int_value, sizeof(int_value), &out_len);
\end{verbatim}

When \var{action} is \code{CS_GET} the method returns a tuple
containing the Sybase result code and a value.  When a string value is
requested an optional \var{item} argument can be passed which defaults
to \code{CS_UNUSED}.

The return result depends upon the value of the \var{type} argument.

\begin{longtable}{l|l|l}
\var{type} & need item? & return values \\
\hline
\code{CS_12HOUR}     & no  & \code{status, bool} \\
\code{CS_DT_CONVFMT} & no  & \code{status, int} \\
\code{CS_MONTH}      & yes & \code{status, string} \\
\code{CS_SHORTMONTH} & yes & \code{status, string} \\
\code{CS_DAYNAME}    & yes & \code{status, string} \\
\code{CS_DATEORDER}  & no  & \code{status, string} \\
\end{longtable}

The Sybase-CT \function{cs_dt_info()} function is called like this:

\begin{verbatim}
/* bool value */
status = cs_dt_info(ctx, CS_GET, locale, type, CS_UNUSED,
                    &bool_value, sizeof(bool_value), &out_len);

/* int value */
status = cs_dt_info(ctx, CS_GET, locale, type, CS_UNUSED,
                    &int_value, sizeof(int_value), &out_len);

/* string value */
status = cs_dt_info(ctx, CS_GET, locale, type, item,
                    str_buff, sizeof(str_buff), &buff_len);
\end{verbatim}
\end{methoddesc}

\begin{methoddesc}[CS_LOCALE]{cs_loc_drop}{}
Calls the Sybase \function{cs_loc_drop()} function like this:

\begin{verbatim}
status = cs_loc_drop(ctx, locale);
\end{verbatim}

Returns the Sybase result code.

This method will be automatically called when the \class{CS_LOCALE}
object is deleted.  Applications do not need to call the method.
\end{methoddesc}

\begin{methoddesc}[CS_LOCALE]{cs_locale}{action, type \optional{, value}}
Load the object with localisation values or retrieves the locale name
previously used to load the object.

When \var{action} is \code{CS_SET} a string \var{value} argument
must be supplied and the method returns the Sybase result code.  The
Sybase-CT \function{cs_locale()} function is called like this:

\begin{verbatim}
status = cs_locale(ctx, CS_SET, locale, type, value, CS_NULLTERM, NULL);
\end{verbatim}

The recognised values for \var{type} are:

\begin{longtable}{l}
\var{type} \\
\hline
\code{CS_LC_COLLATE}  \\
\code{CS_LC_CTYPE}    \\
\code{CS_LC_MESSAGE}  \\
\code{CS_LC_MONETARY} \\
\code{CS_LC_NUMERIC}  \\
\code{CS_LC_TIME}     \\
\code{CS_LC_ALL}      \\
\code{CS_SYB_LANG}    \\
\code{CS_SYB_CHARSET} \\
\code{CS_SYB_SORTORDER} \\
\code{CS_SYB_COLLATE}   \\
\code{CS_SYB_LANG_CHARSET} \\
\code{CS_SYB_TIME}     \\
\code{CS_SYB_MONETARY} \\
\code{CS_SYB_NUMERIC}  \\
\end{longtable}

When \var{action} is \code{CS_GET} the method returns a tuple
containing the Sybase result code and a locale name.  The
Sybase-CT \function{cs_locale()} function is called like this:

\begin{verbatim}
status = cs_locale(ctx, CS_GET, locale, type, str_buff, sizeof(str_buff), &str_len);
\end{verbatim}
\end{methoddesc}

\subsection{CS_CONNECTION Objects}

Calling the \method{ct_con_alloc()} method of a \class{CS_CONTEXT}
object will create a \class{CS_CONNECTION} object.  When the
\class{CS_CONNECTION} object is deallocated the Sybase
\function{ct_con_drop()} function will be called for the connection.

\class{CS_CONNECTION} objects have the following interface:

\begin{memberdesc}[CS_CONNECTION]{ctx}
This is a read only reference to the parent \class{CS_CONTEXT} object.
This prevents the context from being dropped while the connection
still exists.
\end{memberdesc}

\begin{memberdesc}[CS_CONNECTION]{strip}
An integer which controls right whitespace stripping of \code{char}
columns.  The default value is zero.
\end{memberdesc}

\begin{memberdesc}[CS_CONNECTION]{debug}
An integer which controls printing of debug messages to the debug file
established by the \function{set_debug()} function.  The default value
is inherited from the \code{CS_CONTEXT} object.
\end{memberdesc}

\begin{methoddesc}[CS_CONNECTION]{ct_diag}{operation \optional{, \ldots}}
Manage Sybase error messages for the connection.

When \var{operation} is \code{CS_INIT} a single argument is accepted
and the Sybase result code is returned.  The Sybase
\function{ct_diag()} function is called like this:

\begin{verbatim}
status = ct_diag(conn, CS_INIT, CS_UNUSED, CS_UNUSED, NULL);
\end{verbatim}

When \var{operation} is \code{CS_MSGLIMIT} two additional arguments
are expected; \var{type} and \var{num}.  The Sybase result code is
returned.  The Sybase \function{ct_diag()} function is called like this:

\begin{verbatim}
status = ct_diag(conn, CS_MSGLIMIT, type, CS_UNUSED, &num);
\end{verbatim}

When \var{operation} is \code{CS_CLEAR} an additional \var{type}
argument is accepted and the Sybase result code is returned.  The
Sybase \function{ct_diag()} function is called like this:

\begin{verbatim}
status = ct_diag(conn, CS_CLEAR, type, CS_UNUSED, NULL);
\end{verbatim}

When \var{operation} is \code{CS_GET} two additional arguments are
expected; \var{type} and \var{index}.  A tuple is returned which
contains the Sybase result code and the requested \class{CS_SERVERMSG}
or \class{CS_CLIENTMSG} message.  \code{None} is returned as the
message object when the result code is not \code{CS_SUCCEED}.  The
Sybase \function{ct_diag()} function is called like this:

\begin{verbatim}
status = ct_diag(conn, CS_GET, type, index, &msg);
\end{verbatim}

When \var{operation} is \code{CS_STATUS} an additional \var{type}
argument is accepted.  A tuple is returned which contains the Sybase
result code and the number of messages available for retrieval.  The
Sybase \function{ct_diag()} function is called like this:

\begin{verbatim}
status = ct_diag(conn, CS_STATUS, type, CS_UNUSED, &num);
\end{verbatim}

When \var{operation} is \code{CS_EED_CMD} two additional arguments are
expected; \var{type} and \var{index}.  A tuple is returned which
contains the Sybase result code and a \class{CS_COMMAND} object which
is used to retrieve extended error data.  The Sybase
\function{ct_diag()} function is called like this:

\begin{verbatim}
status = ct_diag(conn, CS_EED_CMD, type, index, &eed);
\end{verbatim}

The following will retrieve and print all messages from a connection.

\begin{verbatim}
def print_msgs(conn, type):
    status, num_msgs = conn.ct_diag(CS_STATUS, type)
    if status != CS_SUCCEED:
        return
    for i in range(num_msgs):
        status, msg = conn.ct_diag(CS_GET, type, i + 1)
        if status != CS_SUCCEED:
            continue
        for attr in dir(msg):
            print '%s: %s' % (attr, getattr(msg, attr))

def print_all_msgs(conn):
    print_msgs(conn, CS_SERVERMSG_TYPE)
    print_msgs(conn, CS_CLIENTMSG_TYPE)
    conn.ct_diag(CS_CLEAR, CS_ALLMSG_TYPE)
\end{verbatim}
\end{methoddesc}

\begin{methoddesc}[CS_CONNECTION]{ct_cancel}{type}
Calls the Sybase \function{ct_cancel()} function and returns the
Sybase result code.  The Sybase \function{ct_cancel()} function is
called like this:

\begin{verbatim}
status = ct_cancel(conn, NULL, type);
\end{verbatim}
\end{methoddesc}

\begin{methoddesc}[CS_CONNECTION]{ct_connect}{\optional{server \code{= None}}}
Calls the Sybase \function{ct_connect()} function and returns the
Sybase result code.  The Sybase \function{ct_connect()} function is
called like this:

\begin{verbatim}
status = ct_connect(conn, server, CS_NULLTERM);
\end{verbatim}

When no \var{server} argument is supplied the Sybase
\function{ct_connect()} function is called like this:

\begin{verbatim}
status = ct_connect(conn, NULL, 0);
\end{verbatim}
\end{methoddesc}

\begin{methoddesc}[CS_CONNECTION]{ct_cmd_alloc}{}
Allocates and returns a new \class{CS_COMMAND} object which is used to
send commands over the connection.  Calls the Sybase-CT
\function{ct_callback()} function like this:

\begin{verbatim}
status = ct_cmd_alloc(conn, &cmd);
\end{verbatim}

The result is a tuple containing the Sybase result code and a new
instance of the \class{CS_COMMAND} class. \code{None} is returned as
the \class{CS_COMMAND} object when the result code is not
\code{CS_SUCCEED}.
\end{methoddesc}

\begin{methoddesc}[CS_CONNECTION]{blk_alloc}{\optional{version \code{= BLK_VERSION_100}}}
Allocates and returns a new \class{CS_BLKDESC} object which is used to
perform bulkcopy over the connection.  Calls the Sybase
\function{blk_alloc()} function like this:

\begin{verbatim}
status = blk_alloc(conn, version, &blk);
\end{verbatim}

The result is a tuple containing the Sybase result code and a new
instance of the \class{CS_BLKDESC} class. \code{None} is returned as
the \class{CS_BLKDESC} object when the result code is not
\code{CS_SUCCEED}.
\end{methoddesc}

\begin{methoddesc}[CS_CONNECTION]{ct_close}{\optional{option \code{ = CS_UNUSED}}}
Calls the Sybase \function{ct_close()} function like this:

\begin{verbatim}
status = ct_close(conn, option);
\end{verbatim}

Returns the Sybase result code.
\end{methoddesc}

\begin{methoddesc}[CS_CONNECTION]{ct_con_drop}{}
Calls the Sybase \function{ct_con_drop()} function like this:

\begin{verbatim}
status = ct_con_drop(conn);
\end{verbatim}

Returns the Sybase result code.

This method will be automatically called when the \class{CS_CONNECTION}
object is deleted.  Applications do not need to call the method.
\end{methoddesc}

\begin{methoddesc}[CS_CONNECTION]{ct_con_props}{action, property \optional{, value}}
Sets, retrieves and clears properties of the connection object.

When \var{action} is \code{CS_SET} a compatible \var{value} argument
must be supplied and the method returns the Sybase result code.  The
Sybase-CT \function{ct_con_props()} function is called like this:

\begin{verbatim}
/* boolean property value */
status = ct_con_props(conn, CS_SET, property, &bool_value, CS_UNUSED, NULL);

/* int property value */
status = ct_con_props(conn, CS_SET, property, &int_value, CS_UNUSED, NULL);

/* string property value */
status = ct_con_props(conn, CS_SET, property, str_value, CS_NULLTERM, NULL);
\end{verbatim}

When \var{action} is \code{CS_GET} the method returns a tuple
containing the Sybase result code and the property value.  The
Sybase-CT \function{ct_con_props()} function is called like this:

\begin{verbatim}
/* boolean property value */
status = ct_con_props(conn, CS_GET, property, &bool_value, CS_UNUSED, NULL);

/* int property value */
status = ct_con_props(conn, CS_GET, property, &int_value, CS_UNUSED, NULL);

/* string property value */
status = ct_con_props(conn, CS_GET, property, str_buff, sizeof(str_buff), &buff_len);
\end{verbatim}

When \var{action} is \code{CS_CLEAR} the method returns the Sybase
result code.  The Sybase-CT \function{ct_con_props()} function is
called like this:

\begin{verbatim}
status = ct_con_props(conn, CS_CLEAR, property, NULL, CS_UNUSED, NULL);
\end{verbatim}

The recognised properties are:

\begin{longtable}{l|l}
\var{property} & type \\
\hline
\code{CS_ANSI_BINDS}           & \code{bool} \\
\code{CS_ASYNC_NOTIFS}         & \code{bool} \\
\code{CS_BULK_LOGIN}           & \code{bool} \\
\code{CS_CHARSETCNV}           & \code{bool} \\
\code{CS_CONFIG_BY_SERVERNAME} & \code{bool} \\
\code{CS_DIAG_TIMEOUT}         & \code{bool} \\
\code{CS_DISABLE_POLL}         & \code{bool} \\
\code{CS_DS_COPY}              & \code{bool} \\
\code{CS_DS_EXPANDALIAS}       & \code{bool} \\
\code{CS_DS_FAILOVER}          & \code{bool} \\
\code{CS_EXPOSE_FMTS}          & \code{bool} \\
\code{CS_EXTERNAL_CONFIG}      & \code{bool} \\
\code{CS_EXTRA_INF}            & \code{bool} \\
\code{CS_HIDDEN_KEYS}          & \code{bool} \\
\code{CS_LOGIN_STATUS}         & \code{bool} \\
\code{CS_NOCHARSETCNV_REQD}    & \code{bool} \\
\code{CS_SEC_APPDEFINED}       & \code{bool} \\
\code{CS_SEC_CHALLENGE}        & \code{bool} \\
\code{CS_SEC_CHANBIND}         & \code{bool} \\
\code{CS_SEC_CONFIDENTIALITY}  & \code{bool} \\
\code{CS_SEC_DATAORIGIN}       & \code{bool} \\
\code{CS_SEC_DELEGATION}       & \code{bool} \\
\code{CS_SEC_DETECTREPLAY}     & \code{bool} \\
\code{CS_SEC_DETECTSEQ}        & \code{bool} \\
\code{CS_SEC_ENCRYPTION}       & \code{bool} \\
\code{CS_SEC_INTEGRITY}        & \code{bool} \\
\code{CS_SEC_MUTUALAUTH}       & \code{bool} \\
\code{CS_SEC_NEGOTIATE}        & \code{bool} \\
\code{CS_SEC_NETWORKAUTH}      & \code{bool} \\

\code{CS_CON_STATUS}           & \code{int} \\
\code{CS_LOOP_DELAY}           & \code{int} \\
\code{CS_RETRY_COUNT}          & \code{int} \\
\code{CS_NETIO}                & \code{int} \\
\code{CS_TEXTLIMIT}            & \code{int} \\
\code{CS_DS_SEARCH}            & \code{int} \\
\code{CS_DS_SIZELIMIT}         & \code{int} \\
\code{CS_DS_TIMELIMIT}         & \code{int} \\
\code{CS_ENDPOINT}             & \code{int} \\
\code{CS_PACKETSIZE}           & \code{int} \\
\code{CS_SEC_CREDTIMEOUT}      & \code{int} \\
\code{CS_SEC_SESSTIMEOUT}      & \code{int} \\

\code{CS_APPNAME}              & \code{string} \\
\code{CS_HOSTNAME}             & \code{string} \\
\code{CS_PASSWORD}             & \code{string} \\
\code{CS_SERVERNAME}           & \code{string} \\
\code{CS_USERNAME}             & \code{string} \\
\code{CS_TDS_VERSION}          & \code{string} \\
\code{CS_DS_DITBASE}           & \code{string} \\
\code{CS_DS_PASSWORD}          & \code{string} \\
\code{CS_DS_PRINCIPAL}         & \code{string} \\
\code{CS_DS_PROVIDER}          & \code{string} \\
\code{CS_SEC_KEYTAB}           & \code{string} \\
\code{CS_SEC_MECHANISM}        & \code{string} \\
\code{CS_SEC_SERVERPRINCIPAL}  & \code{string} \\
\code{CS_TRANSACTION_NAME}     & \code{string} \\

\code{CS_LOC_PROP}             & \code{CS_LOCALE} \\

\code{CS_EED_CMD}              & \code{CS_COMMAND} \\
\end{longtable}

For an explanation of the property values and get/set/clear semantics
please refer to the Sybase documentation.

The following will allocate a connection from a library context,
initialise the connection for in-line message handling, and connect to
the named server using the specified username and password.

\begin{verbatim}
def connect_db(ctx, server, user, passwd):
    status, conn = ctx.ct_con_alloc()
    if status != CS_SUCCEED:
        raise CSError(ctx, 'ct_con_alloc')
    if conn.ct_diag(CS_INIT) != CS_SUCCEED:
        raise CTError(conn, 'ct_diag')
    if conn.ct_con_props(CS_SET, CS_USERNAME, user) != CS_SUCCEED:
        raise CTError(conn, 'ct_con_props CS_USERNAME')
    if conn.ct_con_props(CS_SET, CS_PASSWORD, passwd) != CS_SUCCEED:
        raise CTError(conn, 'ct_con_props CS_PASSWORD')
    if conn.ct_connect(server) != CS_SUCCEED:
        raise CTError(conn, 'ct_connect')
    return conn
\end{verbatim}
\end{methoddesc}

\begin{methoddesc}[CS_CONNECTION]{ct_options}{action, option \optional{, value}}
Sets, retrieves and clears server query processing options for
connection.

When \var{action} is \code{CS_SET} a compatible \var{value} argument
must be supplied and the method returns the Sybase result code.  The
Sybase-CT \function{ct_options()} function is called like this:

\begin{verbatim}
/* bool option value */
status = ct_options(conn, CS_SET, option, &bool_value, CS_UNUSED, NULL);

/* int option value */
status = ct_options(conn, CS_SET, option, &int_value, CS_UNUSED, NULL);

/* string option value */
status = ct_options(conn, CS_SET, option, str_value, CS_NULLTERM, NULL);

/* locale option value */
status = ct_options(conn, CS_SET, option, locale, CS_UNUSED, NULL);
\end{verbatim}

When \var{action} is \code{CS_GET} the method returns a tuple
containing the Sybase result code and the option value.  The
Sybase-CT \function{ct_options()} function is called like this:

\begin{verbatim}
/* bool option value */
status = ct_options(conn, CS_GET, option, &bool_value, CS_UNUSED, NULL);

/* int option value */
status = ct_options(conn, CS_GET, option, &int_value, CS_UNUSED, NULL);

/* string option value */
status = ct_options(conn, CS_GET, option, str_buff, sizeof(str_buff), &buff_len);
\end{verbatim}

When \var{action} is \code{CS_CLEAR} the method returns the Sybase
result code.  The Sybase-CT \function{ct_options()} function is called
like this:

\begin{verbatim}
status = ct_options(conn, CS_CLEAR, option, NULL, CS_UNUSED, NULL);
\end{verbatim}

The recognised options are:

\begin{longtable}{l|l}
\var{option} & type \\
\hline
\code{CS_OPT_ANSINULL}       & \code{bool} \\
\code{CS_OPT_ANSIPERM}       & \code{bool} \\
\code{CS_OPT_ARITHABORT}     & \code{bool} \\
\code{CS_OPT_ARITHIGNORE}    & \code{bool} \\
\code{CS_OPT_CHAINXACTS}     & \code{bool} \\
\code{CS_OPT_CURCLOSEONXACT} & \code{bool} \\
\code{CS_OPT_FIPSFLAG}       & \code{bool} \\
\code{CS_OPT_FORCEPLAN}      & \code{bool} \\
\code{CS_OPT_FORMATONLY}     & \code{bool} \\
\code{CS_OPT_GETDATA}        & \code{bool} \\
\code{CS_OPT_NOCOUNT}        & \code{bool} \\
\code{CS_OPT_NOEXEC}         & \code{bool} \\
\code{CS_OPT_PARSEONLY}      & \code{bool} \\
\code{CS_OPT_QUOTED_IDENT}   & \code{bool} \\
\code{CS_OPT_RESTREES}       & \code{bool} \\
\code{CS_OPT_SHOWPLAN}       & \code{bool} \\
\code{CS_OPT_STATS_IO}       & \code{bool} \\
\code{CS_OPT_STATS_TIME}     & \code{bool} \\
\code{CS_OPT_STR_RTRUNC}     & \code{bool} \\
\code{CS_OPT_TRUNCIGNORE}    & \code{bool} \\

\code{CS_OPT_DATEFIRST}      & \code{int} \\
\code{CS_OPT_DATEFORMAT}     & \code{int} \\
\code{CS_OPT_ISOLATION}      & \code{int} \\
\code{CS_OPT_ROWCOUNT}       & \code{int} \\
\code{CS_OPT_TEXTSIZE}       & \code{int} \\

\code{CS_OPT_AUTHOFF}        & \code{string} \\
\code{CS_OPT_AUTHON}         & \code{string} \\
\code{CS_OPT_CURREAD}        & \code{string} \\
\code{CS_OPT_CURWRITE}       & \code{string} \\
\code{CS_OPT_IDENTITYOFF}    & \code{string} \\
\code{CS_OPT_IDENTITYON}     & \code{string} \\
\end{longtable}

For an explanation of the option values and get/set/clear semantics
please refer to the Sybase documentation.
\end{methoddesc}

\subsection{CS_COMMAND Objects}

Calling the \method{ct_cmd_alloc()} method of a \class{CS_CONNECTION}
object will create a \class{CS_COMMAND} object.  When the
\class{CS_COMMAND} object is deallocated the Sybase
\function{ct_cmd_drop()} function will be called for the command.

\class{CS_COMMAND} objects have the following interface:

\begin{memberdesc}[CS_COMMAND]{is_eed}
A read only integer which indicates when the \class{CS_COMMAND} object
is an extended error data command structure.
\end{memberdesc}

\begin{memberdesc}[CS_COMMAND]{conn}
This is a read only reference to the parent \class{CS_CONNECTION}
object.  This prevents the connection from being dropped while the
command still exists.
\end{memberdesc}

\begin{memberdesc}[CS_COMMAND]{strip}
An integer which controls right whitespace stripping of \code{char}
columns.  The default value is inherited from the parent connection
when the command is created.
\end{memberdesc}

\begin{memberdesc}[CS_COMMAND]{debug}
An integer which controls printing of debug messages to the debug file
established by the \function{set_debug()} function.  The default value
is inherited from the parent connection when the command is created.
\end{memberdesc}

\begin{methoddesc}[CS_COMMAND]{ct_bind}{num, datafmt}
Calls the Sybase-CT \function{ct_bind()} function and returns a tuple
containing the Sybase result code and a \class{DataBuf} object which
is used to retrieve data from the column identified by \var{num}.
\code{None} is returned as the \class{DataBuf} object when the result
code is not \code{CS_SUCCEED}.  The Sybase-CT \function{ct_bind()}
function is called like this:

\begin{verbatim}
status = ct_bind(cmd, num, &datafmt, databuf->buff, databuf->copied, databuf->indicator);
\end{verbatim}

See the description of the \method{ct_describe()} method for an
example of how to use this method in Python.
\end{methoddesc}

\begin{methoddesc}[CS_COMMAND]{ct_cancel}{type}
Calls the Sybase \function{ct_cancel()} function like this:

\begin{verbatim}
status = ct_cancel(NULL, cmd, type);
\end{verbatim}

Returns the Sybase result code.
\end{methoddesc}

\begin{methoddesc}[CS_COMMAND]{ct_cmd_drop}{}
Calls the Sybase-CT \function{ct_cmd_drop()} function like this:

\begin{verbatim}
status = ct_cmd_drop(cmd);
\end{verbatim}

Returns the Sybase result code.

This method will be automatically called when the \class{CS_COMMAND}
object is deleted.  Applications do not need to call the method.
\end{methoddesc}

\begin{methoddesc}[CS_COMMAND]{ct_command}{type \optional{, \ldots}}
Calls the Sybase-CT \function{ct_command()} function and returns the
result code.  The \var{type} argument determines the type and number
of additional arguments.

When \var{type} is \code{CS_LANG_CMD} the method must be called like
this:

\code{ct_command(CS_LANG_CMD, \var{sql_text} \optional{, \var{option \code{= CS_UNUSED}}})}

Then the Sybase-CT \function{ct_command()} function is called like
this:

\begin{verbatim}
status = ct_command(cmd, CS_LANG_CMD, sql_text, CS_NULLTERM, option);
\end{verbatim}

When \var{type} is \code{CS_RPC_CMD} the method must be called like
this:

\code{ct_command(CS_RPC_CMD, \var{proc_name} \optional{, \var{option \code{= CS_UNUSED}}})}

Then the Sybase-CT \function{ct_command()} function is called like
this:

\begin{verbatim}
status = ct_command(cmd, CS_RPC_CMD, proc_name, CS_NULLTERM, option);
\end{verbatim}

When \var{type} is \code{CS_MSG_CMD} the method must be called like
this:

\code{ct_command(CS_MSG_CMD, \var{msg_num})}

Then the Sybase-CT \function{ct_command()} function is called like
this:

\begin{verbatim}
status = ct_command(cmd, CS_MSG_CMD, &msg_num, CS_UNUSED, CS_UNUSED);
\end{verbatim}

When \var{type} is \code{CS_PACKAGE_CMD} the method must be called like
this:

\code{ct_command(CS_PACKAGE_CMD, \var{pkg_name})}

Then the Sybase-CT \function{ct_command()} function is called like
this:

\begin{verbatim}
status = ct_command(cmd, CS_PACKAGE_CMD, pkg_name, CS_NULLTERM, CS_UNUSED);
\end{verbatim}

When \var{type} is \code{CS_SEND_DATA_CMD} the method must be called like
this:

\code{ct_command(CS_SEND_DATA_CMD)}

Then the Sybase-CT \function{ct_command()} function is called like
this:

\begin{verbatim}
status = ct_command(cmd, CS_SEND_DATA_CMD, NULL, CS_UNUSED, CS_COLUMN_DATA);
\end{verbatim}

For an explanation of the argument semantics please refer to the
Sybase documentation.
\end{methoddesc}

\begin{methoddesc}[CS_COMMAND]{ct_cursor}{type \optional{, \ldots}}
Calls the Sybase \function{ct_cursor()} function and returns the
result code.  The \var{type} argument determines the type and number
of additional arguments.

When \var{type} is \code{CS_CURSOR_DECLARE} the method must be called like
this:

\code{ct_cursor(CS_CURSOR_DECLARE, \var{cursor_id}, \var{sql_text} \optional{, \var{option \code{= CS_UNUSED}}})}

Then the Sybase-CT \function{ct_cursor()} function is called like
this:

\begin{verbatim}
status = ct_cursor(cmd, CS_CURSOR_DECLARE, cursor_id, CS_NULLTERM, sql_text, CS_NULLTERM, option);
\end{verbatim}

When \var{type} is \code{CS_CURSOR_UPDATE} the method must be called like
this:

\code{ct_cursor(CS_CURSOR_UPDATE, \var{table_name}, \var{sql_text} \optional{, \var{option \code{= CS_UNUSED}}})}

Then the Sybase-CT \function{ct_cursor()} function is called like
this:

\begin{verbatim}
status = ct_cursor(cmd, CS_CURSOR_UPDATE, table_name, CS_NULLTERM, sql_text, CS_NULLTERM, option);
\end{verbatim}

When \var{type} is \code{CS_CURSOR_OPTION} the method must be called like
this:

\code{ct_cursor(CS_CURSOR_OPTION \optional{, \var{option \code{= CS_UNUSED}}})}

Then the Sybase-CT \function{ct_cursor()} function is called like
this:

\begin{verbatim}
status = ct_cursor(cmd, CS_CURSOR_OPTION, NULL, CS_UNUSED, NULL, CS_UNUSED, option);
\end{verbatim}

When \var{type} is \code{CS_CURSOR_OPEN} the method must be called like
this:

\code{ct_cursor(CS_CURSOR_OPEN \optional{, \var{option \code{= CS_UNUSED}}})}

Then the Sybase-CT \function{ct_cursor()} function is called like
this:

\begin{verbatim}
status = ct_cursor(cmd, CS_CURSOR_OPEN, NULL, CS_UNUSED, NULL, CS_UNUSED, option);
\end{verbatim}

When \var{type} is \code{CS_CURSOR_CLOSE} the method must be called like
this:

\code{ct_cursor(CS_CURSOR_CLOSE \optional{, \var{option \code{= CS_UNUSED}}})}

Then the Sybase-CT \function{ct_cursor()} function is called like
this:

\begin{verbatim}
status = ct_cursor(cmd, CS_CURSOR_CLOSE, NULL, CS_UNUSED, NULL, CS_UNUSED, option);
\end{verbatim}

When \var{type} is \code{CS_CURSOR_ROWS} the method must be called like
this:

\code{ct_cursor(CS_CURSOR_ROWS, \var{num_rows})}

Then the Sybase-CT \function{ct_cursor()} function is called like
this:

\begin{verbatim}
status = ct_cursor(cmd, CS_CURSOR_ROWS, NULL, CS_UNUSED, NULL, CS_UNUSED, num_rows);
\end{verbatim}

When \var{type} is \code{CS_CURSOR_DELETE} the method must be called like
this:

\code{ct_cursor(CS_CURSOR_DELETE, \var{table_name})}

Then the Sybase-CT \function{ct_cursor()} function is called like
this:

\begin{verbatim}
status = ct_cursor(cmd, CS_CURSOR_DELETE, table_name, CS_NULLTERM, NULL, CS_UNUSED, CS_UNUSED);
\end{verbatim}

When \var{type} is \code{CS_CURSOR_DEALLOC} the method must be called like
this:

\code{ct_cursor(CS_CURSOR_DEALLOC)}

Then the Sybase-CT \function{ct_cursor()} function is called like
this:

\begin{verbatim}
status = ct_cursor(cmd, CS_CURSOR_DEALLOC, NULL, CS_UNUSED, NULL, CS_UNUSED, CS_UNUSED);
\end{verbatim}

For an explanation of the argument semantics please refer to the
Sybase documentation.

The \texttt{cursor_sel.py}, \texttt{cursor_upd.py}, and
\texttt{dynamic_cur.py} example programs contain examples of this
function.
\end{methoddesc}

\begin{methoddesc}[CS_COMMAND]{ct_data_info}{action, \ldots}
Sets and retrieves column IO descriptors.

When \var{action} is \code{CS_SET} the method must be called like
this:

\code{ct_data_info(CS_SET, \var{iodesc})}

Returns the Sybase result code.  The Sybase-CT
\function{ct_data_info()} function is called like this:

\begin{verbatim}
status = ct_data_info(cmd, CS_SET, CS_UNUSED, &iodesc);
\end{verbatim}

When \var{action} is \code{CS_GET} the method must be called like
this:

\code{ct_data_info(CS_SET, \var{num})}

Returns a tuple containing the Sybase result code and an
\code{CS_IODESC} object.  If the result code is not \code{CS_SUCCEED}
then \code{None} is returned as the \code{CS_IODESC} object.  The
Sybase-CT \function{ct_data_info()} function is called like this:

\begin{verbatim}
status = ct_data_info(cmd, CS_GET, num, &iodesc);
\end{verbatim}

For an explanation of the argument semantics please refer to the
Sybase documentation.

The \texttt{mult_text.py} example program contains examples of this
function.
\end{methoddesc}

\begin{methoddesc}[CS_COMMAND]{ct_describe}{num}
Calls the Sybase \function{ct_describe()} function and returns a tuple
containing the Sybase result code and a \class{CS_DATAFMT} object
which describes the column identified by \var{num}. \code{None} is
returned as the \class{CS_DATAFMT} object when the result code is not
\code{CS_SUCCEED}.

The Sybase-CT \function{ct_describe()} function is called like this:

\begin{verbatim}
status = ct_describe(cmd, num, &datafmt);
\end{verbatim}

The following constructs a list of buffers for retrieving a number of
rows from a command object.

\begin{verbatim}
def row_bind(cmd, count = 1):
    status, num_cols = cmd.ct_res_info(CS_NUMDATA)
    if status != CS_SUCCEED:
        raise 'ct_res_info'
    bufs = []
    for i in range(num_cols):
        status, fmt = cmd.ct_describe(i + 1)
        if status != CS_SUCCEED:
            raise 'ct_describe'
        fmt.count = count
        status, buf = cmd.ct_bind(i + 1, fmt)
        if status != CS_SUCCEED:
            raise 'ct_bind'
        bufs.append(buf)
    return bufs
\end{verbatim}
\end{methoddesc}

\begin{methoddesc}[CS_COMMAND]{ct_dynamic}{type, \ldots}
Calls the Sybase \function{ct_dynamic()} function and returns the
result code.  The \var{type} argument determines the type and number
of additional arguments.

When \var{type} is \code{CS_CURSOR_DECLARE} the method must be called
like this:

\code{ct_dynamic(CS_CURSOR_DECLARE, \var{dyn_id}, \var{cursor_id})}

Then the Sybase-CT \function{ct_dynamic()} function is called like
this:

\begin{verbatim}
status = ct_dynamic(cmd, CS_CURSOR_DECLARE, dyn_id, CS_NULLTERM, cursor_id, CS_NULLTERM);
\end{verbatim}

When \var{type} is \code{CS_DEALLOC} the method must be called like
this:

\code{ct_dynamic(CS_DEALLOC, \var{dyn_id})}

Then the Sybase-CT \function{ct_dynamic()} function is called like
this:

\begin{verbatim}
status = ct_dynamic(cmd, CS_DEALLOC, dyn_id, CS_NULLTERM, NULL, CS_UNUSED);
\end{verbatim}

When \var{type} is \code{CS_DESCRIBE_INPUT} the method must be called
like this:

\code{ct_dynamic(CS_DESCRIBE_INPUT, \var{dyn_id})}

Then the Sybase-CT \function{ct_dynamic()} function is called like
this:

\begin{verbatim}
status = ct_dynamic(cmd, CS_DESCRIBE_INPUT, dyn_id, CS_NULLTERM, NULL, CS_UNUSED);
\end{verbatim}

When \var{type} is \code{CS_DESCRIBE_OUTPUT} the method must be called
like this:

\code{ct_dynamic(CS_DESCRIBE_OUTPUT, \var{dyn_id})}

Then the Sybase-CT \function{ct_dynamic()} function is called like
this:

\begin{verbatim}
status = ct_dynamic(cmd, CS_DESCRIBE_OUTPUT, dyn_id, CS_NULLTERM, NULL, CS_UNUSED);
\end{verbatim}

When \var{type} is \code{CS_EXECUTE} the method must be called like
this:

\code{ct_dynamic(CS_EXECUTE, \var{dyn_id})}

Then the Sybase-CT \function{ct_dynamic()} function is called like
this:

\begin{verbatim}
status = ct_dynamic(cmd, CS_EXECUTE, dyn_id, CS_NULLTERM, NULL, CS_UNUSED);
\end{verbatim}

When \var{type} is \code{CS_EXEC_IMMEDIATE} the method must be called
like this:

\code{ct_dynamic(CS_EXEC_IMMEDIATE, \var{sql_text})}

Then the Sybase-CT \function{ct_dynamic()} function is called like
this:

\begin{verbatim}
status = ct_dynamic(cmd, CS_EXEC_IMMEDIATE, NULL, CS_UNUSED, sql_text, CS_NULLTERM);
\end{verbatim}

When \var{type} is \code{CS_EXECUTE} the method must be called like
this:

\code{ct_dynamic(CS_PREPARE, \var{dyn_id}, \var{sql_text})}

Then the Sybase-CT \function{ct_dynamic()} function is called like
this:

\begin{verbatim}
status = ct_dynamic(cmd, CS_PREPARE, dyn_id, CS_NULLTERM, sql_text, CS_NULLTERM);
\end{verbatim}

For an explanation of the argument semantics please refer to the
Sybase documentation.

The \texttt{dynamic_cur.py}, and \texttt{dynamic_ins.py} example
programs contain examples of this function.
\end{methoddesc}

\begin{methoddesc}[CS_COMMAND]{ct_fetch}{}
Calls the Sybase \function{ct_fetch()} function and returns a tuple
containing the Sybase result code and the number of rows read (for
array binding).

The Sybase-CT \function{ct_fetch()} function is called like this:

\begin{verbatim}
status = ct_fetch(cmd, CS_UNUSED, CS_UNUSED, CS_UNUSED, &rows_read);
\end{verbatim}
\end{methoddesc}

\begin{methoddesc}[CS_COMMAND]{ct_get_data}{num, databuf}
Calls the Sybase \function{ct_get_data()} function and returns a tuple
containing the Sybase result code and the length of the data for item
number \var{num} which was read into the \class{DataBuf} object in the
\var{databuf} argument.

The Sybase-CT \function{ct_get_data()} function is called like this:

\begin{verbatim}
status = ct_get_data(cmd, num, databuf->buff, databuf->fmt.maxlength, databuf->copied);
\end{verbatim}

The following will retrieve the contents of a BLOB column:

\begin{verbatim}
def get_blob_column(cmd, col):
    fmt = CS_DATAFMT()
    fmt.maxlength = 32768
    buf = DataBuf(fmt)
    parts = []
    while 1:
        status, count = cmd.ct_get_data(col, buf)
        if count:
            parts.append(buf[0])
        if status != CS_SUCCEED:
            break
    return string.join(parts, '')
\end{verbatim}
\end{methoddesc}

\begin{methoddesc}[CS_COMMAND]{ct_param}{param}
Calls the Sybase \function{ct_param()} function and returns the Sybase
result code.

The \var{param} argument is usually an instance of the \class{DataBuf}
class.  A \class{CS_DATAFMT} object can be used in a cursor declare
context to define the format of the host variable.

When \var{param} is a \class{DataBuf} the Sybase-CT
\function{ct_param()} function is called like this:

\begin{verbatim}
status = ct_param(cmd, &databuf->fmt, databuf->buff, databuf->copied[0], databuf->indicator[0]);
\end{verbatim}

When \var{param} is a \class{CS_DATAFMT} the Sybase-CT
\function{ct_param()} function is called like this:

\begin{verbatim}
status = ct_param(cmd, &datafmt, NULL, CS_UNUSED, CS_UNUSED);
\end{verbatim}

The semantics of the \class{CS_DATAFMT} attributes are quite complex.
Please refer to the Sybase documentation.
\end{methoddesc}

\begin{methoddesc}[CS_COMMAND]{ct_res_info}{type}
Calls the Sybase \function{ct_res_info()} function.  The return result
depends upon the value of the \var{type} argument.

\begin{longtable}{l|l}
\var{type} & return values \\
\hline
\code{CS_BROWSE_INFO}   & \code{status, bool} \\
\code{CS_CMD_NUMBER}    & \code{status, int} \\
\code{CS_MSGTYPE}       & \code{status, int} \\
\code{CS_NUM_COMPUTES}  & \code{status, int} \\
\code{CS_NUMDATA}       & \code{status, int} \\
\code{CS_NUMORDER_COLS} & \code{status, int} \\
\code{CS_ORDERBY_COLS}  & \code{status, list of int} \\
\code{CS_ROW_COUNT}     & \code{status, int} \\
\code{CS_TRANS_STATE}   & \code{status, int} \\
\end{longtable}

Depending on type the Sybase-CT \function{ct_res_info()} function is
called like this:

\begin{verbatim}
status = ct_res_info(cmd, CS_BROWSE_INFO, &bool_val, CS_UNUSED, NULL);

status = ct_res_info(cmd, CS_MSGTYPE, &ushort_val, CS_UNUSED, NULL);

status = ct_res_info(cmd, CS_CMD_NUMBER, &int_val, CS_UNUSED, NULL);

status = ct_res_info(cmd, CS_NUM_COMPUTES, &int_val, CS_UNUSED, NULL);

status = ct_res_info(cmd, CS_NUMDATA, &int_val, CS_UNUSED, NULL);

status = ct_res_info(cmd, CS_NUMORDERCOLS, &int_val, CS_UNUSED, NULL);

status = ct_res_info(cmd, CS_ROW_COUNT, &int_val, CS_UNUSED, NULL);

status = ct_res_info(cmd, CS_TRANS_STATE, &int_val, CS_UNUSED, NULL);

status = ct_res_info(cmd, CS_NUMORDERCOLS, &int_val, CS_UNUSED, NULL);
status = ct_res_info(cmd, CS_ORDERBY_COLS, col_nums, sizeof(*col_nums) * int_val, NULL);
\end{verbatim}
\end{methoddesc}

\begin{methoddesc}[CS_COMMAND]{ct_results}{}
Calls the Sybase \function{ct_results()} function and returns a tuple
containing the Sybase result code and the result type returned by the
Sybase function.

The Sybase-CT \function{ct_results()} function is called like this:

\begin{verbatim}
status = ct_results(cmd, &result);
\end{verbatim}
\end{methoddesc}

\begin{methoddesc}[CS_COMMAND]{ct_send}{}
Calls the Sybase \function{ct_send()} function and returns the Sybase
result code.

The Sybase-CT \function{ct_send()} function is called like this:

\begin{verbatim}
status = ct_send(cmd);
\end{verbatim}
\end{methoddesc}

\begin{methoddesc}[CS_COMMAND]{ct_send_data}{databuf}
Calls the Sybase \function{ct_send_data()} function and returns the
Sybase result code.  The \var{databuf} argument must be a
\class{DataBuf} object.

The Sybase-CT \function{ct_send_data()} function is called like this:

\begin{verbatim}
status = ct_send_data(cmd, databuf->buff, databuf->copied[0]);
\end{verbatim}
\end{methoddesc}

\begin{methoddesc}[CS_COMMAND]{ct_setparam}{databuf}
Calls the Sybase \function{ct_setparam()} function and returns the
Sybase result code.  The \var{databuf} argument must be a
\class{DataBuf} object.

The Sybase-CT \function{ct_setparam()} function is called like this:

\begin{verbatim}
status = ct_setparam(cmd, &databuf->fmt, databuf->buff, databuf->copied, databuf->indicator);
\end{verbatim}
\end{methoddesc}

\subsection{CS_CLIENTMSG Objects}

\class{CS_CLIENTMSG} objects are a very thing wrapper around the
Sybase \code{CS_CLIENTMSG} structure.  They have the following
read only attributes:

\begin{tabular}{l|l}
attribute & type \\
\hline
\code{severity}  & \code{int} \\
\code{msgnumber} & \code{int} \\
\code{msgstring} & \code{string} \\
\code{osnumber}  & \code{int} \\
\code{osstring}  & \code{string} \\
\code{status}    & \code{int} \\
\code{sqlstate}  & \code{string} \\
\end{tabular}

\subsection{CS_SERVERMSG Objects}

\class{CS_SERVERMSG} objects are a very thing wrapper around the
Sybase \code{CS_SERVERMSG} structure.  They have the following
read only attributes:

\begin{tabular}{l|l}
attribute & type \\
\hline
\code{msgnumber} & \code{int} \\
\code{state}     & \code{int} \\
\code{severity}  & \code{int} \\
\code{text}      & \code{string} \\
\code{svrname}   & \code{string} \\
\code{proc}      & \code{string} \\
\code{line}      & \code{int} \\
\code{status}    & \code{int} \\
\code{sqlstate}  & \code{string} \\
\end{tabular}

\subsection{CS_DATAFMT Objects}

\class{CS_DATAFMT} objects are a very thing wrapper around the
Sybase \code{CS_DATAFMT} structure.  They have the following
attributes:

\begin{tabular}{l|l}
attribute & type \\
\hline
\code{name}      & \code{string} \\
\code{datatype}  & \code{int} \\
\code{format}    & \code{int} \\
\code{maxlength} & \code{int} \\
\code{scale}     & \code{int} \\
\code{precision} & \code{int} \\
\code{status}    & \code{int} \\
\code{count}     & \code{int} \\
\code{usertype}  & \code{int} \\
\code{strip}     & \code{int} \\
\end{tabular}

The \code{strip} attribute is an extension of the Sybase
\code{CS_DATAFMT} structure.  Please refer to the \class{DataBuf}
documentation.

\class{CS_DATAFMT} structures are mostly used to create
\class{DataBuf} objects for sending data to and receiving data from
the server.

A \class{CS_DATAFMT} object created via the \function{CS_DATAFMT()}
constructor will have the following values:

\begin{tabular}{l|l}
attribute & value \\
\hline
\code{name}      & \code{'$\backslash$0'} \\
\code{datatype}  & \code{CS_CHAR_TYPE} \\
\code{format}    & \code{CS_FMT_NULLTERM} \\
\code{maxlength} & \code{1} \\
\code{scale}     & \code{0} \\
\code{precision} & \code{0} \\
\code{status}    & \code{0} \\
\code{count}     & \code{0} \\
\code{usertype}  & \code{0} \\
\code{strip}     & \code{0} \\
\end{tabular}

You will almost certainly need to provide new values for some of the
attributes before you use the object.

A \class{CS_DATAFMT} object created as a return value from the
\function{ct_bind()} function will be ready to use for creating a
\class{DataBuf} object.

\subsection{DataBuf Objects}

\class{DataBuf} objects manage buffers which are used to hold data to
be sent to and received from the server.

\class{DataBuf} objects contain an embedded Sybase \code{CS_DATAFMT}
structure and allocated buffers suitable for binding the contained
data to Sybase-CT API functions.

When constructed from native Python or Sybase data types a buffer is
created for a single value.  When created using a \class{CS_DATAFMT}
object the \code{count} attribute is used to allocate buffers suitable
for array binding.  A \code{count} of zero is treated the same as
\code{1}.

The \class{DataBuf} objects have the same attributes as a
\class{CS_DATAFMT} object but the attributes which describe the memory
are read only and cannot be modified.

\begin{tabular}{l|l|l}
attribute & type & read only? \\
\hline
\code{name}      & \code{string} & no \\
\code{datatype}  & \code{int}    & yes \\
\code{format}    & \code{int}    & no \\
\code{maxlength} & \code{int}    & yes \\
\code{scale}     & \code{int}    & yes \\
\code{precision} & \code{int}    & yes \\
\code{status}    & \code{int}    & no \\
\code{count}     & \code{int}    & yes \\
\code{usertype}  & \code{int}    & yes \\
\code{strip}     & \code{int}    & no \\
\end{tabular}

In addition the \class{DataBuf} object behaves like a fixed length
mutable sequence.

Adapted from \texttt{Sybase.py}, this is how you create a set of
buffers suitable for retrieving a number of rows from the server:

\begin{verbatim}
def row_bind(cmd, count = 1):
    status, num_cols = cmd.ct_res_info(CS_NUMDATA)
    if status != CS_SUCCEED:
        raise 'ct_res_info'
    bufs = []
    for i in range(num_cols):
        status, fmt = cmd.ct_describe(i + 1)
        if status != CS_SUCCEED:
            raise 'ct_describe'
        fmt.count = count
        status, buf = cmd.ct_bind(i + 1, fmt)
        if status != CS_SUCCEED:
            raise 'ct_bind'
        bufs.append(buf)
    return bufs
\end{verbatim}

Then once the rows have been fetched, this is how you extract the data
from the buffers:

\begin{verbatim}
def fetch_rows(cmd, bufs):
    rows = []
    status, rows_read = cmd.ct_fetch()
    if status == CS_SUCCEED:
        for i in range(rows_read):
            row = []
            for buf in bufs:
                row.append(buf[i])
            rows.append(tuple(row))
    return rows
\end{verbatim}

To send a parameter to a dynamic SQL command or a stored procedure you
are likely to create a \class{DataBuf} object directly from the value
you wish to send.  For example:

\begin{verbatim}
if cmd.ct_command(CS_RPC_CMD, 'sp_help', CS_NO_RECOMPILE) != CS_SUCCEED:
    raise 'ct_command'
buf = DataBuf('sysobjects')
buf.status = CS_INPUTVALUE
if cmd.ct_param(buf)  != CS_SUCCEED:
    raise 'ct_param'
if cmd.ct_send() != CS_SUCCEED:
    raise 'ct_send'
\end{verbatim}

Note that it is your responsibility to make sure that the buffers are
not deallocated before you have finished using them.  If you are not
careful you will get a segmentation fault.

\subsection{CS_IODESC Objects}

\class{CS_IODESC} objects are a very thing wrapper around the
Sybase \code{CS_IODESC} structure.  They have the following
attributes:

\begin{tabular}{l|l}
attribute & type \\
\hline
\code{iotype}        & \code{int} \\
\code{datatype}      & \code{int} \\
\code{usertype}      & \code{int} \\
\code{total_txtlen}  & \code{int} \\
\code{offset}        & \code{int} \\
\code{log_on_update} & \code{int} \\
\code{name}          & \code{string} \\
\code{timestamp}     & \code{binary} \\
\code{textptr}       & \code{binary} \\
\end{tabular}

These objects are created either by calling the
\method{ct_data_info()} method of a \class{CS_COMMAND} object, or by
calling the \function{CS_IODESC} constructor.

\subsection{CS_BLKDESC Objects}

Calling the \method{blk_alloc()} method of a \class{CS_CONNECTION}
object will create a \class{CS_BLKDESC} object.  When the
\class{CS_BLKDESC} object is deallocated the Sybase
\function{blk_drop()} function will be called for the command.

\class{CS_BLKDESC} objects have the following interface:

\begin{methoddesc}[CS_BLKDESC]{blk_bind}{num, databuf}
Calls the Sybase \function{blk_bind()} function and returns the Sybase
result code.  The Sybase-CT \function{blk_bind()} function is called
like this:

\begin{verbatim}
status = blk_bind(blk, num, &datafmt, buffer->buff, buffer->copied, buffer->indicator);
\end{verbatim}
\end{methoddesc}

\begin{methoddesc}[CS_BLKDESC]{blk_describe}{num}
Calls the Sybase \function{blk_describe()} function and returns a
tuple containing the Sybase result code and a \class{CS_DATAFMT}
object which describes the column identified by \var{num}. \code{None}
is returned as the \class{CS_DATAFMT} object when the result code is
not \code{CS_SUCCEED}.

The Sybase \function{blk_describe()} function is called like this:

\begin{verbatim}
status = blk_describe(blk, num, &datafmt);
\end{verbatim}
\end{methoddesc}

\begin{methoddesc}[CS_BLKDESC]{blk_done}{type}
Calls the Sybase \function{blk_done()} function and returns a tuple
containing the Sybase result code and the number of rows copied in the
current batch.

The Sybase \function{blk_done()} function is called like this:

\begin{verbatim}
status = blk_done(blk, type, &num_rows);
\end{verbatim}
\end{methoddesc}

\begin{methoddesc}[CS_BLKDESC]{blk_drop}{}
Calls the Sybase \function{blk_drop()} function and returns the Sybase
result code.

The Sybase \function{blk_drop()} function is called like this:

\begin{verbatim}
status = blk_drop(blk);
\end{verbatim}

This method will be automatically called when the \class{CS_BLKDESC}
object is deleted.  Applications do not need to call the method.
\end{methoddesc}

\begin{methoddesc}[CS_BLKDESC]{blk_init}{direction, table}
Calls the Sybase \function{blk_init()} function and returns the Sybase
result code.

The Sybase \function{blk_init()} function is called like this:

\begin{verbatim}
status = blk_init(blk, direction, table, CS_NULLTERM);
\end{verbatim}
\end{methoddesc}

\begin{methoddesc}[CS_BLKDESC]{blk_props}{action, property \optional{, value}}
Sets, retrieves and clears properties of the bulk descriptor object.

When \var{action} is \code{CS_SET} a compatible \var{value} argument
must be supplied and the method returns the Sybase result code.  The
Sybase \function{blk_props()} function is called like this:

\begin{verbatim}
/* boolean property value */
status = blk_props(blk, CS_SET, property, &bool_value, CS_UNUSED, NULL);

/* int property value */
status = blk_props(blk, CS_SET, property, &int_value, CS_UNUSED, NULL);

/* numeric property value */
status = blk_props(blk, CS_SET, property, &numeric_value, CS_UNUSED, NULL);
\end{verbatim}

When \var{action} is \code{CS_GET} the method returns a tuple
containing the Sybase result code and the property value.  The Sybase
\function{blk_props()} function is called like this:

\begin{verbatim}
/* boolean property value */
status = blk_props(blk, CS_GET, property, &bool_value, CS_UNUSED, NULL);

/* int property value */
status = blk_props(blk, CS_GET, property, &int_value, CS_UNUSED, NULL);

/* numeric property value */
status = blk_props(blk, CS_GET, property, &numeric_value, CS_UNUSED, NULL);
\end{verbatim}

When \var{action} is \code{CS_CLEAR} the method returns the Sybase
result code.  The Sybase \function{blk_props()} function is called
like this:

\begin{verbatim}
status = blk_props(blk, CS_CLEAR, property, NULL, CS_UNUSED, NULL);
\end{verbatim}

The recognised properties are:

\begin{longtable}{l|l}
\var{property} & type \\
\hline
\code{BLK_IDENTITY}        & \code{bool} \\
\code{BLK_NOAPI_CHK}       & \code{bool} \\
\code{BLK_SENSITIVITY_LBL} & \code{bool} \\
\code{ARRAY_INSERT}        & \code{bool} \\
\code{BLK_SLICENUM}        & \code{int} \\
\code{BLK_IDSTARTNUM}      & \code{numeric} \\
\end{longtable}

For an explanation of the property values and get/set/clear semantics
please refer to the Sybase documentation.
\end{methoddesc}

\begin{methoddesc}[CS_BLKDESC]{blk_rowxfer}{}
Calls the Sybase \function{blk_rowxfer()} function and returns the
Sybase result code.

The Sybase \function{blk_rowxfer()} function is called like this:

\begin{verbatim}
status = blk_rowxfer(blk);
\end{verbatim}
\end{methoddesc}

\begin{methoddesc}[CS_BLKDESC]{blk_rowxfer_mult}{\optional{row_count}}
Calls the Sybase \function{blk_rowxfer_mult()} function and returns a
tuple containing the Sybase result code and the number of rows
transferred.

The Sybase \function{blk_rowxfer_mult()} function is called like this:

\begin{verbatim}
status = blk_rowxfer_mult(blk, &row_count);
\end{verbatim}
\end{methoddesc}

\begin{methoddesc}[CS_BLKDESC]{blk_textxfer}{\optional{str}}
Calls the Sybase \function{blk_textxfer()} function.  Depending on the
direction of the bulkcopy (established via the \method{blk_init()}
method) the method expects different arguments.

When \code{direction} \code{CS_BLK_IN} the \var{str} argument must be
supplied and method returns the Sybase result code.

The Sybase \function{blk_textxfer()} function is called like this:

\begin{verbatim}
status = blk_textxfer(blk, str, str_len, NULL);
\end{verbatim}

When \code{direction} \code{CS_BLK_OUT} the \var{str} argument must
not be present and method returns a tuple containing the Sybase result
code and a string.

The Sybase \function{blk_textxfer()} function is called like this:

\begin{verbatim}
status = blk_textxfer(blk, buff, sizeof(buff), &out_len);
\end{verbatim}
\end{methoddesc}

A simplistic program to bulkcopy a table from one server to another
server follows:

The first section contains the code to display client and server
messages in case of failure.

\begin{verbatim}
import sys
from sybasect import *

def print_msgs(conn, type):
    status, num_msgs = conn.ct_diag(CS_STATUS, type)
    if status != CS_SUCCEED:
        return
    for i in range(num_msgs):
        status, msg = conn.ct_diag(CS_GET, type, i + 1)
        if status != CS_SUCCEED:
            continue
        for attr in dir(msg):
            sys.stderr.write('%s: %s\n' % (attr, getattr(msg, attr)))

def die(conn, func):
    sys.stderr.write('%s failed!\n' % func)
    print_msgs(conn, CS_SERVERMSG_TYPE)
    print_msgs(conn, CS_CLIENTMSG_TYPE)
    sys.exit(1)
\end{verbatim}

The next section is fairly constant for all CT library programs.  A
library context is allocated and connections established.  The only
thing which is unique to bulk copy operations is setting the
\code{CS_BULK_LOGIN} option on the connection.

\begin{verbatim}
def init_db():
    status, ctx = cs_ctx_alloc()
    if status != CS_SUCCEED:
        raise 'cs_ctx_alloc'
    if ctx.ct_init(CS_VERSION_100) != CS_SUCCEED:
        raise 'ct_init'
    return ctx

def connect_db(ctx, server, user, passwd):
    status, conn = ctx.ct_con_alloc()
    if status != CS_SUCCEED:
        raise 'ct_con_alloc'
    if conn.ct_diag(CS_INIT) != CS_SUCCEED:
        die(conn, 'ct_diag')
    if conn.ct_con_props(CS_SET, CS_USERNAME, user) != CS_SUCCEED:
        die(conn, 'ct_con_props CS_USERNAME')
    if conn.ct_con_props(CS_SET, CS_PASSWORD, passwd) != CS_SUCCEED:
        die(conn, 'ct_con_props CS_PASSWORD')
    if conn.ct_con_props(CS_SET, CS_BULK_LOGIN, 1) != CS_SUCCEED:
        die(conn, 'ct_con_props CS_BULK_LOGIN')
    if conn.ct_connect(server) != CS_SUCCEED:
        die(conn, 'ct_connect')
    return conn
\end{verbatim}

The next segment allocates bulkcopy descriptors, data buffers, and
binds the data buffers to the bulk copy descriptors.  The same buffers
are used for copying out and copying in - not bad.  Note that for array
binding we need to use loose packing for copy in; hence the line
setting the \code{format} member of \code{Databuf} \code{CS_DATAFMT} to
\code{CS_BLK_ARRAY_MAXLEN}.  Without this the bulkcopy operation
assumes tight packing and the data is corrupted on input.

\begin{verbatim}
def alloc_bcp(conn, dirn, table):
    status, blk = conn.blk_alloc()
    if status != CS_SUCCEED:
        die(conn, 'blk_alloc')
    if blk.blk_init(dirn, table) != CS_SUCCEED:
        die(conn, 'blk_init')
    return blk

def alloc_bufs(bcp, num):
    bufs = []
    while 1:
        status, fmt = bcp.blk_describe(len(bufs) + 1)
        if status != CS_SUCCEED:
            break
        fmt.count = num
        bufs.append(DataBuf(fmt))
    return bufs

def bcp_bind(bcp, bufs):
    for i in range(len(bufs)):
        buf = bufs[i]
        if bcp.direction == CS_BLK_OUT:
            buf.format = 0
        else:
            buf.format = CS_BLK_ARRAY_MAXLEN
        if bcp.blk_bind(i + 1, buf) != CS_SUCCEED:
            die(bcp.conn, 'blk_bind')
\end{verbatim}

This next section actually performs the bulkcopy.  Note that there is
no attempt to deal with BLOB columns.

\begin{verbatim}
def bcp_copy(src, dst, batch_size):
    total = batch = 0
    while 1:
        status, num_rows = src.blk_rowxfer_mult()
        if status == CS_END_DATA:
            break
        if status != CS_SUCCEED:
            die(src, 'blk_rowxfer_mult out')
        status, dummy = dst.blk_rowxfer_mult(num_rows)
        if status != CS_SUCCEED:
            die(src, 'blk_rowxfer_mult in')
        batch = batch + num_rows
        if batch >= batch_size:
            total = total + batch
            batch = 0
            src.blk_done(CS_BLK_BATCH)
            dst.blk_done(CS_BLK_BATCH)
            print 'batch - %d rows transferred' % total

    status, num_rows = src.blk_done(CS_BLK_ALL)
    status, num_rows = dst.blk_done(CS_BLK_ALL)
    return total + batch
\end{verbatim}

Finally the code which drives the whole process.

\begin{verbatim}
ctx = init_db()
src_conn = connect_db(ctx, 'drama', 'sa', '')
dst_conn = connect_db(ctx, 'SYBASE', 'sa', '')
src = alloc_bcp(src_conn, CS_BLK_OUT, 'pubs2.dbo.authors')
dst = alloc_bcp(dst_conn, CS_BLK_IN,  'test.dbo.authors')

bufs = alloc_bufs(src, 5)
bcp_bind(src, bufs)
bcp_bind(dst, bufs)

total = bcp_copy(src, dst, 10)
print 'all done - %d rows transferred' % total
\end{verbatim}

